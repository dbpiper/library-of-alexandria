\documentclass[12pt]{article}
\usepackage{enumitem, extsizes, amsfonts, braket, amsmath, ragged2e, physics, amsthm }

\parindent=0pt
\renewcommand\qedsymbol{Q.E.D.}

\title{Chapter 1 Exercises}
\author{David Piper}

\begin{document}
  \maketitle

  \begin{enumerate}[label=1.\arabic*]
    \item D and e are sets as they are in curly braces.
    The other examples have some items just listed by themselves.
    \item
      \begin{enumerate}[label=(\alph*)]
        \item $\set{x \in S \mid x \in \mathbb{N}}$
        \item $\set{x \in S \mid x \geq 0}$
        \item $\set{x \in S \mid x < 0}$
        \item $\set{x \in S \mid \abs{x} > 1}$
      \end{enumerate}
    \item
      \begin{enumerate}[label=(\alph*)]
        \item $|A|=5$
        \item $|B|=21$
        \item $|C|=50$
        \item $|D|=2$
        \item $|E|=1$
        \item $|E|=2$
      \end{enumerate}
    \item
      \begin{enumerate}[label=(\alph*)]
        \item $A = \set{-3, -2, -1, 0, 1, 2, 3, 4}$
        \item $B = \set{-2, -1, 0, 1, 2}$
        \item $C = \set{1,2,3,4,5,6,7,8,9}$
        \item $D = \set{0, 1}$
        \item $E = \emptyset$
      \end{enumerate}
    \item
      \begin{enumerate}[label=(\alph*)]
        \item $A = \set{x \in \mathbb{Z} \mid x < 0}$
        \item $B = \set{x \in \mathbb{Z} \mid x^2 < 10}$
        \item $C = \set{x \in \mathbb{Z} \mid 0 < x^2 < 5}$
      \end{enumerate}
    \item
      \begin{enumerate}[label=(\alph*)]
        \item $A = \set{2x + 1 \mid x \in \mathbb{Z}} 
        = \set{\dots, -3, -1, 1, 3, 4\dots}$
        \item $B = \set{4n \mid n \in \mathbb{Z}} 
        = \set{\dots, -16, -8, -4, 0, 4, 8, 16, \dots}$
        \item $C = \set{3q + 1 \mid q \in \mathbb{Z}} 
        = \set{\dots, -7, -4, 1, 4, 7\dots}$
      \end{enumerate}
    \item
      \begin{enumerate}[label=(\alph*)]
        \item $A = \set{3x + 2 \mid x \in \mathbb{Z}}$
        \item $B = \set{5x \mid x \in \mathbb{Z}}$
        \item $C = \set{x^3 \mid x \in \mathbb{Z}}$
      \end{enumerate}
    \item
      \begin{enumerate}[label=(\alph*)]
        \item $A =\set{-4,-3,-2,2,3,4}$
        \item $\frac{9}{4},\frac{10}{4}, \text{ and } \frac{21}{8}$
        \item $C =\set{2, \sqrt{2}}$
        \item $D =\set{2}$
        \item $\abs{A}=6,\abs{C}=2, \text{ and } \abs{D}=1$
      \end{enumerate}
    \item $B=\set{5,7,8,10,13}$ so $C=\set{5,8}$
    \item
      \begin{enumerate}[label=(\alph*)]
        \item $A=\set{1},B=\set{1}, \text{ and } C=\set{1, 2}$
        \item $A=\set{1},B=\set{\set{1}, 2}, \text{ and } C=\set{\set{\set{1}, 2}}$
        \item $A=\set{1},B=\set{\set{1}, 2}, \text{ and } C=\set{1, 2}$
      \end{enumerate}
    \item The interval $I$ can just be $\left(a, b\right)$ since the two intervals would be equal and thus subsets.
    If we let $c = \dfrac{b + a}{2}$ then the interval $I$ would be centered at $c$; as it would be the center of the original
    interval.
    \item $A=B=C=D=E$
    \item Skipping this because the problem is pretty straightforward and it's really
    complicated to draw Venn diagrams in LaTeX (aka the benefits are not worth the effort).
    \item
      \begin{enumerate}[label=(\alph*)]
        \item $\mathcal{P}\qty(A)=\set{\emptyset,\set{1},\set{2},\set{1,2}}
        \text{ and } \abs{\mathcal{P}\qty(A)}=4$
        \item $\mathcal{P}\qty(A)=\set{\emptyset,\set{\emptyset},\set{1},\set{\emptyset,1}, \\
        \set{\set{a}}, \set{\emptyset, \set{a}}}, \set{1, \set{a}}, \set{\emptyset, 1, \set{a}}$ \\
         and $\abs{\mathcal{P}\qty(A)}=8$
      \end{enumerate}
    \item $\mathcal{P}\qty(A)=\set{\emptyset, \set{0}, \set{\set{0}}, \set{\emptyset,
      \set{0}}, \set{0, \set{0}}, \set{\emptyset, 0, \set{0}}}$
    \item Find $\mathcal{P}\qty(\mathcal{P}\qty(\set{ 1 }))$ and its cardinality.
      $\mathcal{P}\qty(\set{1})=\set{\emptyset,\set{1}}$ \\
      $\mathcal{P}\qty(\mathcal{P}\qty(\set{1}))=\set{\emptyset, \set{\emptyset}, \set{\set{1}}, \set{\emptyset, \set{1}}}$ \\
      $\abs{\mathcal{P}\qty(\mathcal{P}\qty(\set{1}))}=4$
    \item $\mathcal{P}\qty(A)=\set{\emptyset, \set{0}, \set{\emptyset},
    \set{\set{\emptyset}}, \set{0, \emptyset}, \set{\emptyset, \set{\emptyset}},\set{0, \set{\emptyset}},
    \set{0, \emptyset, \set{\emptyset}}}$ \\
    $\abs{\mathcal{P}\qty(A)}=8$
    \item \begin{math}
      A=\set{\emptyset, 0, \set{\emptyset}, \set{0}} \\
      \mathcal{P}\qty(A)=\{ \emptyset, \set{\emptyset}, \set{0}, \set{\set{\emptyset}}, \set{\set{0}}, \set{\emptyset, 0},\\
      \set{\emptyset, \set{\emptyset}}, \set{\emptyset, \set{0}}, \set{\set{0}, \set{\emptyset, 0, \set{\emptyset}}}, \\
      \set{\emptyset, 0, \set{0}}, \set{\emptyset, 0, \set{0}, \set{\emptyset}} \} \\
    \end{math}
    // I am stopping here this is really tedious, I know I am missing a few sets, but it isn't
    valuable enough to spend more time on. May come back with pen and paper.
    \item
      \begin{enumerate}[label=(\alph*)]
        \item $S=\emptyset$
        \item $S=\set{1}$
        \item $\set{\set{1},\set{2},\set{3},\set{4},\set{5}}$
        \item $\set{1,2,3,4,5}$
      \end{enumerate}
    \item
      \begin{enumerate}[label=(\alph*)]
        \item This statement could be either true or false; we don't have enough
              information. Specifically, if the \emph{only} element of $A$ is $1$, then
              it would be true. However if $A=\set{1, \set{1}}$ then it would be false.
        \item
              \begin{proof}
              Since we know that $A \subset \mathcal{P}\qty(B) \subset C$ and that
              $\abs{A}=2$ then we know that $\abs{\mathcal{P}\qty(B)}$ must be
              less than $\abs{C}$; since a proper subset must have less elements than its proper
              superset. So we have the following inequality $\abs{A} < \abs*{\mathcal{P}\qty(B)} < \abs*{C}$. 

              Since the cardinality of a set must be a nonzero positive integer we know that the smallest possible
              cardinality for $\mathcal{P}\qty(B)$ is $3$. However, we also know that the cardinality of a power set
              $\mathcal{P}\qty(K)$ must be $2^{\abs*{K}}$ for some set $K$. Thus, the cardinality
              of $\mathcal{P}\qty(B)$ cannot be $3$ (the next largest integer after $2$; which is $\abs*{A}$) because
              $\log_2{3}$ is not an integer. However, it \emph{can} be $4$. So the smallest possible cardinality
              of $\mathcal{P}\qty(B)=4$. Since $\abs*{\mathcal{P}\qty(B)} < \abs*{C}$ the smallest possible cardinality of
              $C=5$.
              \end{proof}
        \item
              \begin{proof}
                We are given that $\abs*{B}=\abs*{A}+1$. We also know that the cardinality of a power set of
                a set $K$ must be $2^{\abs*{K}}$. So in this case, $\abs*{\mathcal{P}\qty(A)}=2^{\abs*{A}}$ and
                $\abs*{\mathcal{P}\qty(B)}=2^{\abs*{B}}=2^{\abs*{A+1}}$. From this equation we can observe that regardless
                of the cardinality of $B$ it must always be at least $2^1$ or simply $2$ more than the cardinality of $A$.
              \end{proof}
        \item
              \begin{proof}
                We have four sets: $A, B, C, \text{ and } D$ which are subsets of $\set{1,2,3}$. Furthermore,
                we know that $\abs*{A}=\abs*{B}=\abs*{C}=\abs*{D}=2$. There are only $3$ subsets of $\set{1,2,3}$ with
                a cardinality of $2$, namely $\set{1,2},\set{1,3}, \text{ and } \set{2,3}$. Since we have four sets all of which
                are subsets of $\set{1,2,3}$ with a cardinality of $2$ we must conclude that at least one of them are the same set.
              \end{proof}
      \end{enumerate}

      \item One approach we can take to find a solution that satisfies all of the conditions is
            attempting to minimize the sums of the elements in each set.
            We know from (a) that $A$ must contain $1$, from (b) that $A$ and $C$ must contain $2$, and that $A$ must contain $3$ from (c).
            So we know that the minimum sum of $A$ is $6$. If we choose $3$ to be in $C$ from (c) then the difference in the sums of $A$ and $C$ is exactly $1$
            which satisfies (f). So we'd like to maintain that if we can. Observing that (d) requires $4$ to be in an even number of any of the sets we can
            choose to add it to both $A$ and $C$, thus maintaining the difference of $1$ for their sums. If we then choose to satisfy (e) by adding $5$ to $B$ we 
            have satisfied all of the conditions leaving us with $B=\set{1,5}$.
      \item
            \begin{enumerate}[label=(\alph*)]
              \item $A \cup B = \set{1,5,9,13,3,15}$
              \item $A \cap B = \set{9}$
              \item $A - B = \set{1,5,13}$
              \item $B - A = \set{3,15}$
              \item $\overline{A} = \set{3,7,11,15}$
              \item $A \cap \overline{B} = \set{1,5,13}$
            \end{enumerate}
      \item To solve this problem, first observe that since $\abs{A - B}=3$, there
            must be $3$ elements in $A$ that are not in $B$. Similarly, since $\abs{B - A}=3$
            there must be $3$ elements in $B$ that are not in $A$. Further, since
            $\abs{A \cap B}=3$, there must be $3$ elements that they have in common.

            We may choose any two sets that satisfy these conditions, so for the sake
            of simplicity let us choose $A=\set{1,2,3,7,8,9}$ and $B=\set{4,5,6,7,8,9}$.
            In this case we have that $A-B=\set{1,2,3}$ with a cardinality of $3$ and we
            have $B-A=\set{4,5,6}$ with a cardinality of $3$ and finally we have that
            $A \cap B=\set{7,8,9}$ with a cardinality of $3$. (I did draw Venn diagram, but it is
            on paper).
      \item \begin{proof}
            First observe that since $B-A=C-A$ everything that is in $B$ must be in $C$
            and vice versa except for whatever is in $A$. Further, since $B \neq C$ the
            elements that are shared between $A$ and $B$ must be different than those shared
            between $A$ and $C$. With these facts in hand we may choose any three sets that
            satisfy these conditions. For the sake of simplicity let us choose $A=\set{1,2,3,4}$,
            $B=\set{1,3,5}$, and $C=\set{2,4,5}$. In this case $B-A=\set{5}$ and $C-A=\set{5}$
            thus $B-A=C-A$ and $B \neq C$.
            \end{proof}
  \end{enumerate}
\end{document}