\documentclass[12pt]{article}
\usepackage{
  enumitem,
  extsizes,
  amsfonts,
  braket,
  amsmath,
  ragged2e,
  physics,
  amsthm,
  textcomp,
  gensymb,
  booktabs
}

% Thanks to Heiko Oberdiek for this solution to enumerations with odd numbers only
% found here:
% https://tex.stackexchange.com/questions/184198/skipping-every-even-numbered-item-in-the-enumerate-environment
\makeatletter
\newcommand*{\arabicodd}[1]{%
  \expandafter\@arabicodd\csname c@#1\endcsname
}
\newcommand*{\arabiceven}[1]{%
  \expandafter\@arabiceven\csname c@#1\endcsname
}
\newcommand*{\@arabicodd}[1]{%
  \@arabic{\numexpr(#1)*2-1\relax}%
}
\newcommand*{\@arabiceven}[1]{%
  \@arabic{\numexpr(#1)*2\relax}%
}

\AddEnumerateCounter\arabicodd\@arabicodd{0}
\AddEnumerateCounter\arabiceven\@arabiceven{0}
\makeatother
% end arabic odd/even definition

% Define a closed sqrt function to have a tail
% this more clearly defines the scope of the root
% Source: https://en.wikibooks.org/wiki/LaTeX/Mathematics#Roots
% it renames \sqrt as \oldsqrt
\let\oldsqrt\sqrt
% it defines the new \sqrt in terms of the old one
\def\sqrt{\mathpalette\DHLhksqrt}
\def\DHLhksqrt#1#2{%
\setbox0=\hbox{$#1\oldsqrt{#2\,}$}\dimen0=\ht0
\advance\dimen0-0.2\ht0
\setbox2=\hbox{\vrule height\ht0 depth -\dimen0}%
{\box0\lower0.4pt\box2}}
%

\parindent=0pt
\renewcommand\qedsymbol{Q.E.D.}

\title{Section 2.4 Exercises}
\author{David Piper}

\begin{document}
\maketitle

% The start index is the ith odd number, not the actual value
\begin{enumerate}[label=2.\arabicodd*, start=10]
  \item
        \begin{enumerate}[label=(\alph*)]
          \item $17$ is odd. True.
          \item Either $17$ is even or $19$ is prime. True.
          \item It is the case that $17$ is even and $19$ is prime. False.
          \item If $17$ is even, then $19$ is prime. True.
        \end{enumerate}
  \item Given the statements: $P: \sqrt{2}$ is rational and $Q: \dfrac{22}{7}$ is rational.
        \begin{enumerate}[label=(\alph*)]
          \item If $\sqrt{2}$ is rational, then $\dfrac{22}{7}$ is rational. True.
          \item If $\dfrac{22}{7}$ is rational, then $\sqrt{2}$ is rational. False.
          \item If $\sqrt{2}$ is irrational, then $\dfrac{22}{7}$ is irrational. False.
          \item If $\dfrac{22}{7}$ is irrational, then $\sqrt{2}$ is irrational. True.
        \end{enumerate}
  \item
        \begin{enumerate}[label=(\alph*)]
          \item True.
          \item False.
          \item True.
          \item True.
          \item False.
        \end{enumerate}
  \item
        \begin{enumerate}[label=(\alph*)]
          \item True.
          \item False.
          \item True.
          \item True.
          \item True.
        \end{enumerate}
  \item Don and Cindy went to the lecture. This follows from the fact that Cindy
        can go despite Ben not going (the implication is still true) and if Cindy goes
        then Don must go. Any other combination would have resulted in more than two
        students attending.
  \item Which of the following implies that $P \lor Q$ is false?
        \begin{enumerate}[label=(\alph*)]
          \item In this case $P \lor Q$ must be true because the only
                way that $\qty(\sim P) \lor \qty(\sim Q)$ is false is when both
                $P$ and $Q$ are true.
          \item In this case we cannot know whether $P$ is false or
                $Q$ is true, so $P \lor Q$ can be true without any
                contradictions.
          \item Since both $P$ and $Q$ are false their disjunction must
                also be false.
          \item Since $Q \implies P$ it is possible that $P \lor Q$ is true.
          \item Since the conjunction is false we know that at least one of them
                is false, however it could be the case that one is true so $P \lor Q$
                could still be true.
        \end{enumerate}
\end{enumerate}

\end{document}