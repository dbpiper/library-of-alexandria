\documentclass[12pt]{article}
\usepackage{
  enumitem,
  extsizes,
  amsfonts,
  braket,
  amsmath,
  ragged2e,
  physics,
  amsthm,
  textcomp,
  gensymb,
  booktabs,
}

% Thanks to Heiko Oberdiek for this solution to enumerations with odd numbers only
% found here:
% https://tex.stackexchange.com/questions/184198/skipping-every-even-numbered-item-in-the-enumerate-environment
\makeatletter
\newcommand*{\arabicodd}[1]{%
  \expandafter\@arabicodd\csname c@#1\endcsname
}
\newcommand*{\arabiceven}[1]{%
  \expandafter\@arabiceven\csname c@#1\endcsname
}
\newcommand*{\@arabicodd}[1]{%
  \@arabic{\numexpr(#1)*2-1\relax}%
}
\newcommand*{\@arabiceven}[1]{%
  \@arabic{\numexpr(#1)*2\relax}%
}

\AddEnumerateCounter\arabicodd\@arabicodd{0}
\AddEnumerateCounter\arabiceven\@arabiceven{0}
\makeatother
% end arabic odd/even definition

% Define a closed sqrt function to have a tail
% this more clearly defines the scope of the root
% Source: https://en.wikibooks.org/wiki/LaTeX/Mathematics#Roots
% it renames \sqrt as \oldsqrt
\let\oldsqrt\sqrt
% it defines the new \sqrt in terms of the old one
\def\sqrt{\mathpalette\DHLhksqrt}
\def\DHLhksqrt#1#2{%
\setbox0=\hbox{$#1\oldsqrt{#2\,}$}\dimen0=\ht0
\advance\dimen0-0.2\ht0
\setbox2=\hbox{\vrule height\ht0 depth -\dimen0}%
{\box0\lower0.4pt\box2}}
%


\parindent=0pt
\renewcommand\qedsymbol{Q.E.D.}

\title{Section 2.9 Exercises}
\author{David Piper}

\begin{document}
\maketitle

% The start index is the ith odd number, not the actual value
\begin{enumerate}[label=2.\arabicodd*, start=31]
  \item \begin{enumerate}[label=(\alph*)]
    \item The negation of the open sentence ``Either $x=0$ or $y=0$.'' is
          ``Both $x \neq 0$ and $y \neq 0$.'' by DeMorgan's Law.
    \item The negation of the open sentence ``The integers $a$ and $b$ are both even.''
          is ``At least one of the integers $a$ and $b$ are odd.'' by DeMorgan's Law.
  \end{enumerate}
  \item For a real number $x,$ let $P\qty(x): x^2 =2$ and $Q\qty(x): x = \sqrt{2}.$
        We want to state the negation of the biconditional $P \iff Q$ in words.
        First we need to find a logically equivalent version of the negation of this
        biconditional, which we can do using Theorem 2.25(b). This theorem says
        that $\sim\qty(P \iff Q) \equiv \qty(P \land \qty(\sim Q)) \lor \qty(Q \land \qty(\sim P)).$

        If we state this logically equivalent version in words then we have the
        statement ``Either $x^2=2$ and $x \neq \sqrt{2}$ or $x=\sqrt{2}$ and $x^2 \neq 2.$''
  \item For a natural number $n,$ we are given the compound statement 3n + 4 is odd and 5n - 6 is even.
        We want to find an implication such that its negation is this given compound statement.
        From Theorem 2.25(a) we know that $\sim \qty(P \implies Q) \equiv P \land \qty(\sim Q).$
        So to make our given statements match this form we should first take the
        first statement to be $P\qty(n)$ and then take the negation of the second
        statement to be $Q\qty(n).$

        Now we have two statements $P\qty(n): 3n + 4$ is odd and $Q\qty(n): 5n - 6$ is odd and we
        can restate our given compound statement as $P \land \qty(\sim Q).$ Therefore, by Theorem 2.25(a)
        we can restate this statement in its logically equivalent form of $\sim \qty(P \implies Q).$
        Looking at this we can observe that the implication whose negation gives us the original statement
        is simply $P \implies Q$ which we can write in words as ``If $3n +4$ is odd then $5n - 6$ is odd.''
\end{enumerate}
\end{document}