\documentclass[12pt]{article}
\usepackage{enumitem, extsizes, amsfonts, braket, amsmath, ragged2e, physics, amsthm }

% Thanks to Heiko Oberdiek for this solution to enumerations with odd numbers only
% found here:
% https://tex.stackexchange.com/questions/184198/skipping-every-even-numbered-item-in-the-enumerate-environment
\makeatletter
\newcommand*{\arabicodd}[1]{%
  \expandafter\@arabicodd\csname c@#1\endcsname
}
\newcommand*{\arabiceven}[1]{%
  \expandafter\@arabiceven\csname c@#1\endcsname
}
\newcommand*{\@arabicodd}[1]{%
  \@arabic{\numexpr(#1)*2-1\relax}%
}
\newcommand*{\@arabiceven}[1]{%
  \@arabic{\numexpr(#1)*2\relax}%
}

\AddEnumerateCounter\arabicodd\@arabicodd{0}
\AddEnumerateCounter\arabiceven\@arabiceven{0}
\makeatother

\parindent=0pt
\renewcommand\qedsymbol{Q.E.D.}

\title{Section 2.1 Exercises}
\author{David Piper}

\begin{document}
  \maketitle

  \begin{enumerate}[label=2.\arabicodd*]
    \item
      \begin{enumerate}[label=(\alph*)]
        \item The sentence: the integer $123$ is prime, is a
              valid statement and it has a truth value of False.
        \item The sentence: the integer $0$ is even, is a
              valid statement and it has a truth value of True.
        \item The sentence:  Is $5 \times 2 = 10?$ is not a valid
              statement since it is a question and is not declarative.
        \item The sentence:  $x^2 - 4 = 0$ is an open sentence, however we are
              not given the domain for the variable $x.$ If we assume that $x$
              can be anything in the real numbers then this statement has
              a truth value of True when $x=-2$ or when $x=2$ and it has a
              truth value of False otherwise.
        \item The sentence: multiply $5x + 2$ by $3$, is not a valid statement
              as it is an imperative command.
        \item The sentence:  $5x + 3$ is an odd integer is an open statement
              and if assume that the domain of $x$ is the integers, as seems
              appropriate given that the sentence refers to the integers for
              the range of the function, then it is an open statement over the
              domain of $\mathbb{Z}.$ This statement happens to have a truth
              value of True when x is even and a value of False when x is odd.
              \begin{proof}
                $5x + 3$ is an odd integer when x is an even integer.
                Since x is even it can be rewritten as $x=2k$ for some
                $k \in \mathbb{Z}.$ So we have $5\qty(2k)+3=$
                $10k+3=2\qty(5k)+3=2\qty(5k)+\qty(2\qty(1)+1)=$
                $2\qty(5k)+2+1=2\qty(5k+2)+1.$ If we let $l=5k+2$ we
                can rewrite $2(5k+2) + 1$ as $2l+1$. Since the definition
                of an odd integer is any integer that can be expressed as
                2 times some arbitrary integer plus 1 we must conclude that
                the statement $5x+3$ is an odd integer \emph{is} true when
                x is an even integer.
              \end{proof}
              \begin{proof}
                $5x+3$ is not an odd integer when x is an odd integer.
                For the sake of simplicity and time we can simply prove this
                by finding a single counter example; it is a relatively simple
                matter to establish a more long-form proof, but doing so is
                unnecessary in this case. If we let $x=1$ then we have that
                $5x+3=8$ which is not an odd integer, since we cannot write 8
                as $2$ times some integer $k$ + 1. In fact we can write $8$ as $2$
                times $4$ which is an even integer not an odd integer.
              \end{proof}
        \item The sentence: ``What an impossible question!'' is not a valid
              statement as it is simply an exclamatory sentence.
      \end{enumerate}
  \end{enumerate}
\end{document}