\documentclass[12pt]{article}
\usepackage{enumitem, extsizes, amsfonts, braket, amsmath, ragged2e, physics, amsthm }

% Thanks to Heiko Oberdiek for this solution to enumerations with odd numbers only
% found here:
% https://tex.stackexchange.com/questions/184198/skipping-every-even-numbered-item-in-the-enumerate-environment
\makeatletter
\newcommand*{\arabicodd}[1]{%
  \expandafter\@arabicodd\csname c@#1\endcsname
}
\newcommand*{\arabiceven}[1]{%
  \expandafter\@arabiceven\csname c@#1\endcsname
}
\newcommand*{\@arabicodd}[1]{%
  \@arabic{\numexpr(#1)*2-1\relax}%
}
\newcommand*{\@arabiceven}[1]{%
  \@arabic{\numexpr(#1)*2\relax}%
}

\AddEnumerateCounter\arabicodd\@arabicodd{0}
\AddEnumerateCounter\arabiceven\@arabiceven{0}
\makeatother

\parindent=0pt
\renewcommand\qedsymbol{Q.E.D.}

\title{Section 2.1 Exercises}
\author{David Piper}

\begin{document}
  \maketitle

  \begin{enumerate}[label=2.\arabicodd*]
    \item
      \begin{enumerate}[label=(\alph*)]
        \item The sentence: the integer $123$ is prime, is a
              valid statement and it has a truth value of False.
        \item The sentence: the integer $0$ is even, is a
              valid statement and it has a truth value of True.
        \item The sentence:  Is $5 \times 2 = 10?$ is not a valid
              statement since it is a question and is not declarative.
        \item The sentence:  $x^2 - 4 = 0$ is an open sentence, however we are
              not given the domain for the variable $x.$ If we assume that $x$
              can be anything in the real numbers then this statement has
              a truth value of True when $x=-2$ or when $x=2$ and it has a
              truth value of False otherwise.
        \item The sentence: multiply $5x + 2$ by $3$, is not a valid statement
              as it is an imperative command.
        \item The sentence:  $5x + 3$ is an odd integer is an open statement
              and if assume that the domain of $x$ is the integers, as seems
              appropriate given that the sentence refers to the integers for
              the range of the function, then it is an open statement over the
              domain of $\mathbb{Z}.$ This statement happens to have a truth
              value of True when x is even and a value of False when x is odd.
              \begin{proof}
                $5x + 3$ is an odd integer when x is an even integer.
                Since x is even it can be rewritten as $x=2k$ for some
                $k \in \mathbb{Z}.$ So we have $5\qty(2k)+3=$
                $10k+3=2\qty(5k)+3=2\qty(5k)+\qty(2\qty(1)+1)=$
                $2\qty(5k)+2+1=2\qty(5k+2)+1.$ If we let $l=5k+2$ we
                can rewrite $2(5k+2) + 1$ as $2l+1$. Since the definition
                of an odd integer is any integer that can be expressed as
                2 times some arbitrary integer plus 1 we must conclude that
                the statement $5x+3$ is an odd integer \emph{is} true when
                x is an even integer.
              \end{proof}
              \begin{proof}
                $5x+3$ is not an odd integer when x is an odd integer.
                For the sake of simplicity and time we can simply prove this
                by finding a single counter example; it is a relatively simple
                matter to establish a more long-form proof, but doing so is
                unnecessary in this case. If we let $x=1$ then we have that
                $5x+3=8$ which is not an odd integer, since we cannot write 8
                as $2$ times some integer $k$ + 1. In fact we can write $8$ as $2$
                times $4$ which is an even integer not an odd integer.
              \end{proof}
        \item The sentence: ``What an impossible question!'' is not a valid
              statement as it is simply an exclamatory sentence.
      \end{enumerate}
    \item
      \begin{enumerate}[label=(\alph*)]
        \item The statement: $\emptyset \in \emptyset$ is False as the empty
              set, by definition has no elements and thus cannot contain the
              empty set.
        \item The statement: $\emptyset \in \set{\emptyset}$ is True, we can
              simply observe that the definition of being an element of a set
              means that that item is one of the members of the set. Reading
              off the list of items in the set $\set{\emptyset}$ we can clearly
              see that yes $\emptyset$ is in fact in the set.
        \item The statement: $\set{1, 3} = \set{3, 1}$ is True, we can
              use the definition of set equality to see this. Specifically,
              sets are considered to be equal if they have the same members,
              since both of these sets contain the elements: $1$ and $3$ and
              no other explicit elements they are equal.
        \item The statement: $\emptyset = \set{\emptyset}$ is False, one way
              to think about this is to look at the cardinality of each of these
              two sets. The first set namely $\emptyset$ is a set containing
              $0$ elements, thus having a cardinality of $0$. The second set
              namely $\set{\emptyset}$ is a set with a single element $\emptyset$
              having a cardinality of $1$. So these two sets have different
              elements and even different numbers of elements, so given the
              definition of set quality which we have previously discussed;
              namely the requirement that the sets have the same elements
              these two sets cannot possibly be equal.
        \item The statement: $\emptyset \subset \set{\emptyset}$ is True, we
              can observe that the definition of a subset is that a set meeting
              this criteria must have at least some of the elements of the
              superset and none that the superset doesn't have. In this case
              we have the $\emptyset$ or $\set{}$ which is a subset of \emph{every}
              set by its definition.
        \item The statement: $1 \subseteq \set{1}$ is False, we can use
              the definition of subset to determine this. In order for something
              to be a subset of a set, that thing itself must be a \emph{set}
              in this case the statement is asking if the integer $1$ is a subset
              of the set containing the integer $1$. This is a nonsensical question
              as the integer $1$ is \emph{not} a set so it cannot, by definition,
              be a subset of \emph{any} sets.
      \end{enumerate}
    \item
        Given the open sentence $P\qty(x) : 3x - 2 > 4$ we can first observe
        that $3x - 2 = 4$ when $x=2$, by some basic arithmetic namely $3x-2=4$
        $\rightarrow$ $3x=6$ $\rightarrow$ $x=2.$ This fact will allow us to
        determine the values of x for which this sentence is both true and false.
      \begin{enumerate}[label=(\alph*)]
        \item Given the open sentence $P\qty(x) : 3x - 2 > 4$ we can use
              the fact that $3x-2=4$ when $x=2$ to establish the lower-bound
              for the first value of $x$ that satisfies this inequality namely
              $x=3.$ So the interval for values of $x$ which satisfy this inequality
              is: $\left[3, \infty \right).$
        \item Given the open sentence $P\qty(x) : 3x - 2 > 4$ we can use
            the fact that $3x-2=4$ when $x=2$ to establish the upper-bound
            for the last value of $x$ that does not satisfy this inequality
            namely $x=2.$ So the interval for values of $x$ which do not satisfy
            this inequality is: $\left(-\infty, 2\right].$
      \end{enumerate}
    \item We are given $P\qty(n): n$ and $n+2$ are primes and are told that this
          is an open sentence over the domain $\mathbb{N}.$ Six positive integers
          n that satisfy this open sentence in the domain are: $\set{3,5,11,17,29,41}.$
    \item We are given $S = \set{3,5,7,9}$ and our task is to find some open sentence
          $P\qty(n)$ over the domain $S$ such that $P\qty(n)$ is true for half
          of the integers in $S$ and false for the other half. One such example
          for $P\qty(n)$ is $P\qty(n) : x > 6.$ In this case exactly half of
          the elements of $S$ satisfy this condition, namely $\set{7,9}$ and
          the other half does not: $\set{3,5}.$
  \end{enumerate}
\end{document}