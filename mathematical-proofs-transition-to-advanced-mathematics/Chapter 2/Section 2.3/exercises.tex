\documentclass[12pt]{article}
\usepackage{
  enumitem,
  extsizes,
  amsfonts,
  braket,
  amsmath,
  ragged2e,
  physics,
  amsthm,
  textcomp,
  gensymb,
  booktabs
}

% Thanks to Heiko Oberdiek for this solution to enumerations with odd numbers only
% found here:
% https://tex.stackexchange.com/questions/184198/skipping-every-even-numbered-item-in-the-enumerate-environment
\makeatletter
\newcommand*{\arabicodd}[1]{%
  \expandafter\@arabicodd\csname c@#1\endcsname
}
\newcommand*{\arabiceven}[1]{%
  \expandafter\@arabiceven\csname c@#1\endcsname
}
\newcommand*{\@arabicodd}[1]{%
  \@arabic{\numexpr(#1)*2-1\relax}%
}
\newcommand*{\@arabiceven}[1]{%
  \@arabic{\numexpr(#1)*2\relax}%
}

\AddEnumerateCounter\arabicodd\@arabicodd{0}
\AddEnumerateCounter\arabiceven\@arabiceven{0}
\makeatother
% end arabic odd/even definition

% Define a closed sqrt function to have a tail
% this more clearly defines the scope of the root
% Source: https://en.wikibooks.org/wiki/LaTeX/Mathematics#Roots
% it renames \sqrt as \oldsqrt
\let\oldsqrt\sqrt
% it defines the new \sqrt in terms of the old one
\def\sqrt{\mathpalette\DHLhksqrt}
\def\DHLhksqrt#1#2{%
\setbox0=\hbox{$#1\oldsqrt{#2\,}$}\dimen0=\ht0
\advance\dimen0-0.2\ht0
\setbox2=\hbox{\vrule height\ht0 depth -\dimen0}%
{\box0\lower0.4pt\box2}}
%

\parindent=0pt
\renewcommand\qedsymbol{Q.E.D.}

\title{Section 2.3 Exercises}
\author{David Piper}

\begin{document}
\maketitle

% The start index is the 6th odd number, not the actual value
\begin{enumerate}[label=2.\arabicodd*, start=8]
  \item
        \begin{tabular}{@{}llll@{}}
          \toprule
          P & Q & ${\sim}$Q & P$\land$(${\sim}$Q) \\ \midrule
          T & T & F       & F                            \\
          T & F & T       & T                            \\
          F & T & F       & F                            \\
          F & F & T       & F                            \\ \bottomrule
        \end{tabular}
  \item Given the following statements $P:15$ is odd and
        $Q: 21$ is prime. The following are ways of writing their
        conjunctions and disjunctions in English along with the truth
        values for each combination.
        \begin{enumerate}[label=(\alph*)]
          \item $P \lor Q:$ Either $15$ is odd or $21$ is prime and maybe
                both. True.
          \item $P \land Q:$ It is the case that $15$ is odd and $21$ is
                prime. False.
          \item $\qty({\sim}P) \lor Q:$ Either $15$ is even or $21$ is prime
                and maybe both. False.
          \item $P\land\qty({\sim}Q):$ It is the case that $15$ is odd and
                that $21$ is not prime. True.
        \end{enumerate}
\end{enumerate}
\end{document}