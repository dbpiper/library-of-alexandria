\documentclass[12pt]{article}
\usepackage{
  enumitem,
  extsizes,
  amsfonts,
  braket,
  amsmath,
  ragged2e,
  physics,
  amsthm,
  textcomp,
  gensymb,
  booktabs
}

% Thanks to Heiko Oberdiek for this solution to enumerations with odd numbers only
% found here:
% https://tex.stackexchange.com/questions/184198/skipping-every-even-numbered-item-in-the-enumerate-environment
\makeatletter
\newcommand*{\arabicodd}[1]{%
  \expandafter\@arabicodd\csname c@#1\endcsname
}
\newcommand*{\arabiceven}[1]{%
  \expandafter\@arabiceven\csname c@#1\endcsname
}
\newcommand*{\@arabicodd}[1]{%
  \@arabic{\numexpr(#1)*2-1\relax}%
}
\newcommand*{\@arabiceven}[1]{%
  \@arabic{\numexpr(#1)*2\relax}%
}

\AddEnumerateCounter\arabicodd\@arabicodd{0}
\AddEnumerateCounter\arabiceven\@arabiceven{0}
\makeatother
% end arabic odd/even definition

% Define a closed sqrt function to have a tail
% this more clearly defines the scope of the root
% Source: https://en.wikibooks.org/wiki/LaTeX/Mathematics#Roots
% it renames \sqrt as \oldsqrt
\let\oldsqrt\sqrt
% it defines the new \sqrt in terms of the old one
\def\sqrt{\mathpalette\DHLhksqrt}
\def\DHLhksqrt#1#2{%
\setbox0=\hbox{$#1\oldsqrt{#2\,}$}\dimen0=\ht0
\advance\dimen0-0.2\ht0
\setbox2=\hbox{\vrule height\ht0 depth -\dimen0}%
{\box0\lower0.4pt\box2}}
%

\parindent=0pt
\renewcommand\qedsymbol{Q.E.D.}

\title{Section 2.5 Exercises}
\author{David Piper}

\begin{document}
\maketitle

% The start index is the ith odd number, not the actual value
\begin{enumerate}[label=2.\arabicodd*, start=16]
  \item
    \begin{enumerate}[label=(\alph*)]
      \item Given the open sentences $P\qty(x): \abs{x} = 4$ and $Q\qty(x): x = 4$
            over the domain $S = \set{-4,-3,1,4,5}.$
            We have the following truth values for $P\qty(x) \implies Q\qty(x)$ for each
            $x \in S.$
            \begin{gather*}
              P\qty(-4) \implies Q\qty(-4)\\
              T \implies F\\
              \qq{False.}\\
              P\qty(-3) \implies Q\qty(-3)\\
              F \implies F\\
              \qq{True.}\\
              P\qty(1) \implies Q\qty(1)\\
              F \implies F\\
              \qq{True.}\\
              P\qty(4) \implies Q\qty(4)\\
              T \implies T\\
              \qq{True.}\\
              P\qty(5) \implies Q\qty(5)\\
              F \implies F\\
              \qq{True.}\\
            \end{gather*}
      \item Given the open sentences $P\qty(x): x^{2} = 16$ and $Q\qty(x): \abs{x} = 4$
            over the domain $S = \set{-6,-4,0,3,4,8}.$
            We have the following truth values for $P\qty(x) \implies Q\qty(x)$ for each
            $x \in S.$
            \begin{gather*}
              P\qty(-6) \implies Q\qty(-6)\\
              F \implies F\\
              \qq{True.}\\
              P\qty(-4) \implies Q\qty(-4)\\
              T \implies T\\
              \qq{True.}\\
              P\qty(0) \implies Q\qty(0)\\
              F \implies F\\
              \qq{True.}\\
              P\qty(3) \implies Q\qty(3)\\
              F \implies F\\
              \qq{True.}\\
              P\qty(4) \implies Q\qty(4)\\
              T \implies T\\
              \qq{True.}\\
              P\qty(8) \implies Q\qty(8)\\
              F \implies F\\
              \qq{True.}\\
            \end{gather*}
      \item Given the open sentences $P\qty(x): x > 3$ and $Q\qty(x): 4x - 1 > 12$
            over the domain $S = \set{0,2,3,4,6}.$
            We have the following truth values for $P\qty(x) \implies Q\qty(x)$ for each
            $x \in S.$
            \begin{gather*}
              P\qty(0) \implies Q\qty(0)\\
              F \implies F\\
              \qq{True.}\\
              P\qty(2) \implies Q\qty(2)\\
              F \implies F\\
              \qq{True.}\\
              P\qty(3) \implies Q\qty(3)\\
              F \implies F\\
              \qq{True.}\\
              P\qty(4) \implies Q\qty(4)\\
              T \implies T\\
              \qq{True.}\\
              P\qty(6) \implies Q\qty(6)\\
              T \implies T\\
              \qq{True.}\\
            \end{gather*}
    \end{enumerate}
    \item
      \begin{enumerate}[label=(\alph*)]
        \item Given the open sentences $P\qty(x,y): x^2 - y^2 = 0$ and $Q\qty(x,y): x = y$
              over the domain $S = \set{\qty(1,-1),\qty(3,4),\qty(5,5)}.$
              We have the following truth values for $P\qty(x,y) \implies Q\qty(x,y)$ for each
              given values of $x$ and $y.$
              \begin{gather*}
                P\qty(1,-1) \implies Q\qty(1,-1)\\
                T \implies F\\
                \qq{False.}\\
                P\qty(3,4) \implies Q\qty(3,4)\\
                F \implies F\\
                \qq{True.}\\
                P\qty(5,5) \implies Q\qty(5,5)\\
                T \implies T\\
                \qq{True.}\\
              \end{gather*}
        \item Given the open sentences $P\qty(x,y): \abs{x} = \abs{y}$ and $Q\qty(x,y): x = y$
              over the domain $S = \set{\qty(1,2),\qty(2,-2),\qty(6,6)}.$
              We have the following truth values for $P\qty(x,y) \implies Q\qty(x,y)$ for each
              given values of $x$ and $y.$
              \begin{gather*}
                P\qty(1,2) \implies Q\qty(1,2)\\
                F \implies F\\
                \qq{True.}\\
                P\qty(2,-2) \implies Q\qty(2,-2)\\
                T \implies F\\
                \qq{False.}\\
                P\qty(6,6) \implies Q\qty(6,6)\\
                T \implies T\\
                \qq{True.}\\
              \end{gather*}
        \item Given the open sentences $P\qty(x,y): x^2 + y^2 = 1$ and $Q\qty(x,y): x + y = 1$
              over the domain $S = \set{\qty(1,-1),\qty(-3,4),\qty(0,-1),\qty(1,0)}.$
              We have the following truth values for $P\qty(x,y) \implies Q\qty(x,y)$ for each
              given values of $x$ and $y.$
              \begin{gather*}
                P\qty(1,-1) \implies Q\qty(1,-1)\\
                F \implies F\\
                \qq{True.}\\
                P\qty(-3,4) \implies Q\qty(-3,4)\\
                F \implies T\\
                \qq{True.}\\
                P\qty(0,-1) \implies Q\qty(0,-1)\\
                T \implies F\\
                \qq{False.}\\
                P\qty(1,0) \implies Q\qty(1,0)\\
                T \implies T\\
                \qq{True.}\\
              \end{gather*}
      \end{enumerate}
\end{enumerate}

\end{document}