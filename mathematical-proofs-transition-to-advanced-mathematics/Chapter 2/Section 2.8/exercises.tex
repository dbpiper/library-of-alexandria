\documentclass[12pt]{article}
\usepackage{
  enumitem,
  extsizes,
  amsfonts,
  braket,
  amsmath,
  ragged2e,
  physics,
  amsthm,
  textcomp,
  gensymb,
  booktabs,
}

% Thanks to Heiko Oberdiek for this solution to enumerations with odd numbers only
% found here:
% https://tex.stackexchange.com/questions/184198/skipping-every-even-numbered-item-in-the-enumerate-environment
\makeatletter
\newcommand*{\arabicodd}[1]{%
  \expandafter\@arabicodd\csname c@#1\endcsname
}
\newcommand*{\arabiceven}[1]{%
  \expandafter\@arabiceven\csname c@#1\endcsname
}
\newcommand*{\@arabicodd}[1]{%
  \@arabic{\numexpr(#1)*2-1\relax}%
}
\newcommand*{\@arabiceven}[1]{%
  \@arabic{\numexpr(#1)*2\relax}%
}

\AddEnumerateCounter\arabicodd\@arabicodd{0}
\AddEnumerateCounter\arabiceven\@arabiceven{0}
\makeatother
% end arabic odd/even definition

% Define a closed sqrt function to have a tail
% this more clearly defines the scope of the root
% Source: https://en.wikibooks.org/wiki/LaTeX/Mathematics#Roots
% it renames \sqrt as \oldsqrt
\let\oldsqrt\sqrt
% it defines the new \sqrt in terms of the old one
\def\sqrt{\mathpalette\DHLhksqrt}
\def\DHLhksqrt#1#2{%
\setbox0=\hbox{$#1\oldsqrt{#2\,}$}\dimen0=\ht0
\advance\dimen0-0.2\ht0
\setbox2=\hbox{\vrule height\ht0 depth -\dimen0}%
{\box0\lower0.4pt\box2}}
%


\parindent=0pt
\renewcommand\qedsymbol{Q.E.D.}

\title{Section 2.8 Exercises}
\author{David Piper}

\begin{document}
\maketitle

% The start index is the ith odd number, not the actual value
\begin{enumerate}[label=2.\arabicodd*, start=27]
  \item
        \begin{enumerate}[label=(\alph*)]
          \item
                For the statements $P$ and $Q$ we want to show that $\qty(\sim P) \implies \qty(\sim Q)$
                and $P \implies Q$ are not logically equivalent.
                We start by observing that these statements have the following
                truth table.
                \hfill
                \hfill
                \linebreak
                \linebreak
                \begin{minipage}{\linewidth}
                  \small
                  \begin{tabular}{@{}lllll@{}}
                    \toprule
                    P & Q & $\qty(\sim P) \implies \qty(\sim Q)$ & $P \implies Q$ \\ \midrule
                    T & T & T                                    & T              \\
                    T & F & T                                    & F              \\
                    F & T & F                                    & T              \\
                    F & F & T                                    & T              \\
                    \bottomrule
                  \end{tabular}
                \end{minipage}
                \linebreak
                \linebreak
                \linebreak
                \linebreak
                From this truth table we can observe that the values for the
                statements $\qty(\sim P) \implies \qty(\sim Q)$ and $P \implies Q$
                are not the same in all of their respective rows, which means that
                they are not logically equivalent.
          \item Another implication that is logically equivalent to $\qty(\sim P) \implies \qty(\sim Q)$
                is $Q \implies P.$ To verify this observe that these statements have the following truth table.
                \hfill
                \hfill
                \linebreak
                \linebreak
                \begin{minipage}{\linewidth}
                  \small
                  \begin{tabular}{@{}lllll@{}}
                    \toprule
                    P & Q & $\qty(\sim P) \implies \qty(\sim Q)$ & $Q \implies P$ \\ \midrule
                    T & T & T                                    & T              \\
                    T & F & T                                    & T              \\
                    F & T & F                                    & F              \\
                    F & F & T                                    & T              \\
                    \bottomrule
                  \end{tabular}
                \end{minipage}
                \linebreak
                \linebreak
                \linebreak
                \linebreak
                From this truth table we can observe that the values for the
                statements $\qty(\sim P) \implies \qty(\sim Q)$ and $Q \implies P$
                are the same in all of their respective rows, which means that
                they are logically equivalent.
        \end{enumerate}
  \item \begin{enumerate}[label=(\alph*)]
          \item
                For the statements $P, Q$ and $R$ we want to show that $\qty(P \land Q) \iff P$
                and $P \implies Q$ are logically equivalent.
                We start by observing that these statements have the following
                truth table.
                \hfill
                \hfill
                \linebreak
                \linebreak
                \begin{minipage}{\linewidth}
                  \small
                  \begin{tabular}{@{}lllll@{}}
                    \toprule
                    P & Q & $\qty(P \land Q) \iff P$ & $P \implies Q$ \\ \midrule
                    T & T & T                        & T              \\
                    T & F & F                        & F              \\
                    F & T & T                        & T              \\
                    F & F & T                        & T              \\
                    \bottomrule
                  \end{tabular}
                \end{minipage}
                \linebreak
                \linebreak
                \linebreak
                \linebreak
                From this truth table we can observe that the values for the
                statements $\qty(P \land Q) \iff P$ and $P \implies Q$
                are the same in all of their respective rows, which means that
                they are logically equivalent.
          \item
                For the statements $P, Q$ and $R$ we want to show that $P \implies \qty(Q \lor R)$
                and $\qty(\sim Q) \implies \qty(\qty(\sim P) \lor R)$ are logically equivalent.
                We start by observing that these statements have the following
                truth table.
                \hfill
                \hfill
                \linebreak
                \linebreak
                \begin{minipage}{\linewidth}
                  \small
                  \begin{tabular}{@{}lllllll@{}}
                    \toprule
                    P & Q & R & $Q \lor R$ & $P \implies \qty(Q \lor R)$ & $\qty(\sim P) \lor R$ & $\qty(\sim Q) \implies \qty(\qty(\sim P) \lor R)$ \\ \midrule
                    T & T & T & T          & T                           & T                     & T                                                 \\
                    T & T & F & T          & T                           & F                     & T                                                 \\
                    T & F & T & T          & T                           & T                     & T                                                 \\
                    T & F & F & F          & F                           & F                     & F                                                 \\
                    F & T & T & T          & T                           & T                     & T                                                 \\
                    F & T & F & T          & T                           & T                     & T                                                 \\
                    F & F & T & T          & T                           & T                     & T                                                 \\
                    F & F & F & F          & T                           & T                     & T                                                 \\
                    \bottomrule
                  \end{tabular}
                \end{minipage}
                \linebreak
                \linebreak
                \linebreak
                \linebreak
                From this truth table we can observe that the values for the
                statements $P \implies \qty(Q \lor R)$ and $\qty(\sim Q) \implies \qty(\qty(\sim P) \lor R)$
                are the same in all of their respective rows, which means that
                they are logically equivalent.
        \end{enumerate}
  \item
        For the statements $P, Q$ and $R$ we want to show that $\qty(P \lor Q) \implies R$
        and $\qty(P \implies R) \land \qty(Q \implies R)$ are logically equivalent.
        We start by observing that these statements have the following
        truth table.
        \hfill
        \hfill
        \linebreak
        \linebreak
        \begin{minipage}{\linewidth}
          \small
          \begin{tabular}{@{}lllllll@{}}
            \toprule
            P & Q & R & $P \lor Q \implies R$ & $P \implies R$ & $Q \implies R$ & $\qty(P \implies R) \land \qty(Q \implies R)$ \\ \midrule
            T & T & T & T                     & T              & T              & T                                             \\
            T & T & F & F                     & F              & F              & F                                             \\
            T & F & T & T                     & T              & T              & T                                             \\
            T & F & F & F                     & F              & T              & F                                             \\
            F & T & T & T                     & T              & T              & T                                             \\
            F & T & F & F                     & T              & F              & F                                             \\
            F & F & T & T                     & T              & T              & T                                             \\
            F & F & F & T                     & T              & T              & T                                             \\
            \bottomrule
          \end{tabular}
        \end{minipage}
        \linebreak
        \linebreak
        \linebreak
        \linebreak
        From this truth table we can observe that the values for the
        statements $\qty(P \lor Q) \implies R$
        and $\qty(P \implies R) \land \qty(Q \implies R)$
        are the same in all of their respective rows, which means that
        they are logically equivalent.
    \item \begin{proof}
      We are given five compound statements: $S_1, S_2, S_3, S_4$ and $S_5;$
      which we are also told are all comprised of the same statements $P$
      and $Q$ and they have truth tables which have the same values on the
      first and fourth rows.

      We want to show that at least two of these five statements must be
      logically equivalent given these facts.

      Since the statements have identical truth values in the first and fourth rows
      then we may observe that this acts as a selection of the possible values for
      those rows. That is, for each row generally we may choose any possible value
      of True or False for each cell, but since we know that two rows are identical
      these are already chosen for us.

      Since these are already chosen, we have reduced the possible combinations
      from $2^4$ to $2^2$ for the remaining rows. Since we have five compound
      statements and only $4$ possible different values for the remaining rows
      then we know that at least one of these five must have the same value
      as one of the others.
    \end{proof}
\end{enumerate}
\end{document}