\documentclass[12pt]{article}
\usepackage{
  enumitem,
  extsizes,
  amsfonts,
  braket,
  amsmath,
  ragged2e,
  physics,
  amsthm,
  textcomp,
  gensymb,
  booktabs,
}

% Thanks to Heiko Oberdiek for this solution to enumerations with odd numbers only
% found here:
% https://tex.stackexchange.com/questions/184198/skipping-every-even-numbered-item-in-the-enumerate-environment
\makeatletter
\newcommand*{\arabicodd}[1]{%
  \expandafter\@arabicodd\csname c@#1\endcsname
}
\newcommand*{\arabiceven}[1]{%
  \expandafter\@arabiceven\csname c@#1\endcsname
}
\newcommand*{\@arabicodd}[1]{%
  \@arabic{\numexpr(#1)*2-1\relax}%
}
\newcommand*{\@arabiceven}[1]{%
  \@arabic{\numexpr(#1)*2\relax}%
}

\AddEnumerateCounter\arabicodd\@arabicodd{0}
\AddEnumerateCounter\arabiceven\@arabiceven{0}
\makeatother
% end arabic odd/even definition

% Define a closed sqrt function to have a tail
% this more clearly defines the scope of the root
% Source: https://en.wikibooks.org/wiki/LaTeX/Mathematics#Roots
% it renames \sqrt as \oldsqrt
\let\oldsqrt\sqrt
% it defines the new \sqrt in terms of the old one
\def\sqrt{\mathpalette\DHLhksqrt}
\def\DHLhksqrt#1#2{%
\setbox0=\hbox{$#1\oldsqrt{#2\,}$}\dimen0=\ht0
\advance\dimen0-0.2\ht0
\setbox2=\hbox{\vrule height\ht0 depth -\dimen0}%
{\box0\lower0.4pt\box2}}
%


\parindent=0pt
\renewcommand\qedsymbol{Q.E.D.}

\title{Section 2.6 Exercises}
\author{David Piper}

\begin{document}
\maketitle

% The start index is the ith odd number, not the actual value
\begin{enumerate}[label=2.\arabicodd*, start=18]
  \item Let $P: 18$ is odd and $Q: 25$ is even. Then $P \iff Q$ can be written
        as $18$ is odd if and only if $25$ is even. Furthermore  $P \iff Q$ is true.
  \item For the open sentences $P\qty(x): \abs{x-3} < 1$ and $Q\qty(x): x \in \qty(2,4)$
        over the domain $\mathbb{R}$ the biconditional $P\qty(x) \iff Q\qty(x)$ can be
        stated in a few different ways including the following.
        \begin{enumerate}[label=\arabic*.]
          \item $\abs{x-3}<1$ if and only if $x \in \qty(2,4).$
          \item $\abs{x-3}<1$ is equivalent to $x \in \qty(2,4).$
        \end{enumerate}
  \item For the following open sentences $P\qty(x)$ and $Q\qty(x)$ over a domain
        $S,$ determine all values of $x \in S$ for which the biconditional
        $P\qty(x) \iff Q\qty(x)$ is true.
        \begin{enumerate}[label=(\alph*)]
          \item $P\qty(x): \abs{x}=4; Q\qty(x): x=4; S=\set{-4,-3,1,4,5}.$
                This has the following corresponding truth table.
                \hfill
                \hfill
                \linebreak
                \linebreak
                \begin{minipage}{0.4\linewidth}
                  \begin{tabular}{@{}llllll@{}}
                    \toprule
                    x    & -4 & 3 & 1 & 4 & 5 \\ \midrule
                    P(x) & T  & F & F & T & F \\
                    Q(x) & F  & F & F & T & F \\ \bottomrule
                  \end{tabular}
                \end{minipage}
                \linebreak
                \linebreak
                From this truth table we can observe that $P\qty(x) \iff Q\qty(x)$
                is false when $x=-4$ and true for all other vales of $x \in S.$
          \item $P\qty(x): x \geq 3; Q\qty(x): 4x-1 > 12; S=\set{0,2,3,4,6}.$
                This has the following corresponding truth table.
                \hfill
                \hfill
                \linebreak
                \linebreak
                \begin{minipage}{0.4\linewidth}
                  \begin{tabular}{@{}llllll@{}}
                    \toprule
                    x    & 0 & 2 & 3 & 4 & 6 \\ \midrule
                    P(x) & F & F & T & T & T \\
                    Q(x) & F & F & F & T & T \\ \bottomrule
                  \end{tabular}
                \end{minipage}
                \linebreak
                \linebreak
                From this truth table we can observe that $P\qty(x) \iff Q\qty(x)$
                is false when $x=3$ and true for all other vales of $x \in S.$
          \item $P\qty(x): x^2 = 16; Q\qty(x): x^2 - 4x = 0; S=\set{-6,-4,0,3,4,8}.$
                This has the following corresponding truth table.
                \hfill
                \hfill
                \linebreak
                \linebreak
                \begin{minipage}{0.4\linewidth}
                  \begin{tabular}{@{}lllllll@{}}
                    \toprule
                    x    & -6 & -4 & 0 & 3 & 4 & 8 \\ \midrule
                    P(x) & F  & T  & F & F & T & F \\
                    Q(x) & F  & F  & T & F & T & F \\ \bottomrule
                  \end{tabular}
                \end{minipage}
                \linebreak
                \linebreak
                From this truth table we can observe that $P\qty(x) \iff Q\qty(x)$
                is false when $x=-4$ and when $x=0$ and true for all other vales of $x \in S.$
        \end{enumerate}
  \item The task is to determine all values of $n$ in the domain $S=\set{1,2,3}$
        for which the following is a true statement: A necessary and sufficient
        condition for $\dfrac{n^3+n}{2}$ to be even is that $\dfrac{n^2+n}{2}$
        is odd.

        First we may observe that this question is really a biconditional question
        with two open sentences over the domain $S.$ Given this, we can rewrite
        the problem by saying that we are given two sentences $P\qty(n): \dfrac{n^3+n}{2}$ is even
        and $Q\qty(n): \dfrac{n^2+n}{2}$ is odd over the domain $S=\set{1,2,3}.$

        This has the following truth table.
        \hfill
        \hfill
        \linebreak
        \linebreak
        \begin{minipage}{0.4\linewidth}
          \begin{tabular}{@{}llll@{}}
            \toprule
            n    & 1 & 2 & 3 \\ \midrule
            P(n) & F & F & F \\
            Q(n) & T & T & F \\ \bottomrule
          \end{tabular}
        \end{minipage}
        \linebreak
        \linebreak
        From this truth table we can observe that $P\qty(n) \iff Q\qty(n)$
        is true when $n=3$ and is false for all other vales of $n \in S.$
  \item Given two open sentences $P\qty(n): \dfrac{\qty(n+4)\qty(n+5)}{2}$ is odd
        and $Q\qty(n): 2^{n-2} + 3^{n-2} + 6^{n-2} > \qty(2.5)^{n-1}$ over the domain $S=\set{1,2,3}.$
        We need to determine three distinct elements $a,b,c$ in $S$ such that
        $P\qty(a) \implies Q\qty(a)$ is false, $Q\qty(b) \implies P\qty(b)$ is false
        and $P\qty(c) \iff Q\qty(c)$ is true.

        First let us observe that these statements have the following truth table.
        \hfill
        \hfill
        \linebreak
        \linebreak
        \begin{minipage}{0.4\linewidth}
          \begin{tabular}{@{}llll@{}}
            \toprule
            n    & 1 & 2 & 3 \\ \midrule
            P(n) & T & T & F \\
            Q(n) & F & T & T \\ \bottomrule
          \end{tabular}
        \end{minipage}
        \linebreak
        \linebreak
        From this truth table we can observe that $P\qty(a) \implies Q\qty(a)$ is false
        when $a=1,$ $Q\qty(b) \implies P\qty(b)$ is false when $b=3,$ and
        $P\qty(c) \iff Q\qty(c)$ is true when $c=2.$
  \item Given two open sentences $P\qty(n): 2^n - 1$ is prime
        and $Q\qty(n): n$ is prime over the domain $S=\set{2,3,4,5,6,11}.$
        We need to determine all values of $n \in S$ for which $P\qty(n) \iff Q\qty(n)$
        is a true statement.

        First let us observe that these statements have the following truth table.
        \hfill
        \hfill
        \linebreak
        \linebreak
        \begin{minipage}{0.4\linewidth}
          \begin{tabular}{@{}lllllll@{}}
            \toprule
            n    & 2 & 3 & 4 & 5 & 6 & 11 \\ \midrule
            P(n) & T & T & F & T & F & F  \\
            Q(n) & T & T & F & T & F & T  \\ \bottomrule
          \end{tabular}
        \end{minipage}
        \linebreak
        \linebreak
        From this truth table we can observe that $P\qty(n) \iff Q\qty(n)$ is
        false when $n=11$ and true for all other values of $n \in S.$
\end{enumerate}
\end{document}