\documentclass[12pt]{article}
\usepackage{
  enumitem,
  extsizes,
  amsfonts,
  braket,
  amsmath,
  ragged2e,
  physics,
  amsthm,
  textcomp,
  gensymb,
  booktabs,
  rotating,
}

% Thanks to Heiko Oberdiek for this solution to enumerations with odd numbers only
% found here:
% https://tex.stackexchange.com/questions/184198/skipping-every-even-numbered-item-in-the-enumerate-environment
\makeatletter
\newcommand*{\arabicodd}[1]{%
  \expandafter\@arabicodd\csname c@#1\endcsname
}
\newcommand*{\arabiceven}[1]{%
  \expandafter\@arabiceven\csname c@#1\endcsname
}
\newcommand*{\@arabicodd}[1]{%
  \@arabic{\numexpr(#1)*2-1\relax}%
}
\newcommand*{\@arabiceven}[1]{%
  \@arabic{\numexpr(#1)*2\relax}%
}

\AddEnumerateCounter\arabicodd\@arabicodd{0}
\AddEnumerateCounter\arabiceven\@arabiceven{0}
\makeatother
% end arabic odd/even definition

% Define a closed sqrt function to have a tail
% this more clearly defines the scope of the root
% Source: https://en.wikibooks.org/wiki/LaTeX/Mathematics#Roots
% it renames \sqrt as \oldsqrt
\let\oldsqrt\sqrt
% it defines the new \sqrt in terms of the old one
\def\sqrt{\mathpalette\DHLhksqrt}
\def\DHLhksqrt#1#2{%
\setbox0=\hbox{$#1\oldsqrt{#2\,}$}\dimen0=\ht0
\advance\dimen0-0.2\ht0
\setbox2=\hbox{\vrule height\ht0 depth -\dimen0}%
{\box0\lower0.4pt\box2}}
%


\parindent=0pt
\renewcommand\qedsymbol{Q.E.D.}

\title{Section 2.7 Exercises}
\author{David Piper}

\begin{document}
\maketitle

% The start index is the ith odd number, not the actual value
\begin{enumerate}[label=2.\arabicodd*, start=24]
  \item
        For two statements $P$ and $Q$ we want to show that $\qty(P \land \qty(\sim Q)) \land \qty(P \land Q)$
        and $\qty(P \implies \sim Q) \land \qty(P \land Q)$ are contradictions.
        First let us observe that $\qty(P \land \qty(\sim Q)) \land \qty(P \land Q)$ has the following truth table.
        \hfill
        \hfill
        \hfill
        \linebreak
        \linebreak
        \begin{minipage}{\linewidth}
          \begin{tabular}{@{}llllll@{}}
            \toprule
            P & Q & $\sim Q$ & $P \land \qty(\sim Q)$ & $P \land Q$ & $\qty(P \land \qty(\sim Q)) \land \qty(P \land Q)$ \\ \midrule
            T & T & F        & F                      & T           & F                                                  \\
            T & F & T        & T                      & F           & F                                                  \\
            F & T & F        & F                      & F           & F                                                  \\
            F & F & T        & F                      & F           & F                                                  \\ \bottomrule
          \end{tabular}
        \end{minipage}
        \linebreak
        \linebreak
        From this truth table we can observe that $\qty(P \land \qty(\sim Q)) \land \qty(P \land Q)$ is
        false for all possible values of P and Q, thus it is a contradiction.
        \linebreak
        \linebreak
        Next let's consider the statement $\qty(P \implies \sim Q) \land \qty(P \land Q).$
        We can observe that it has the following truth table.
        \hfill
        \hfill
        \hfill
        \linebreak
        \linebreak
        \begin{minipage}{\linewidth}
          \begin{tabular}{@{}llllll@{}}
            \toprule
            P & Q & $\sim Q$ & $P \implies \sim Q$ & $P \land Q$ & $\qty(P \implies \sim Q) \land \qty(P \land Q)$ \\ \midrule
            T & T & F        & F                   & T           & F                                               \\
            T & F & T        & T                   & F           & F                                               \\
            F & T & F        & T                   & F           & F                                               \\
            F & F & T        & T                   & F           & F                                               \\ \bottomrule
          \end{tabular}
        \end{minipage}
        \linebreak
        \linebreak
        From this truth table we can observe that $\qty(P \implies \sim Q) \land \qty(P \land Q)$ is
        false for all possible values of P and Q, thus it is a contradiction.
  \item
        For the statements $P, Q$ and $R$ we want to show that $\qty(\qty(P \implies Q) \land \qty(Q \implies R)) \implies \qty(P \implies R)$
        is a tautology. First let us observe that these statements have the following
        truth table.
        \hfill
        \hfill
        \hfill
        \linebreak
        \linebreak
        \begin{minipage}{\linewidth}
          \begin{sidewaystable}
            \begin{tabular}{@{}llllllll@{}}
              \toprule
              P & Q & R & $P \implies Q$ & $Q \implies R$ & $\qty(P \implies Q) \land \qty(Q \implies R)$ & $P \implies R$ & $\qty(\qty(P \implies Q) \land \qty(Q \implies R)) \implies \qty(P \implies R)$ \\ \midrule
              % T & T & F        & F                      & T           & F                                                  \\
              % T & F & T        & T                      & F           & F                                                  \\
              % F & T & F        & F                      & F           & F                                                  \\
              % F & F & T        & F                      & F           & F                                                  \\ \bottomrule
            \end{tabular}
          \end{sidewaystable}
        \end{minipage}
        \linebreak
        \linebreak
        From this truth table we can observe that $\qty(P \land \qty(\sim Q)) \land \qty(P \land Q)$ is
        false for all possible values of P and Q, thus it is a contradiction.
\end{enumerate}
\end{document}