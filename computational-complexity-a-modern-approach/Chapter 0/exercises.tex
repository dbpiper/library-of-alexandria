\documentclass[12pt]{article}
\usepackage{enumitem, extsizes, amsfonts, braket, amsmath, ragged2e, physics, amsthm }

% Thanks to Heiko Oberdiek for this solution to enumerations with odd numbers only
% found here:
% https://tex.stackexchange.com/questions/184198/skipping-every-even-numbered-item-in-the-enumerate-environment
\makeatletter
\newcommand*{\arabicodd}[1]{%
  \expandafter\@arabicodd\csname c@#1\endcsname
}
\newcommand*{\arabiceven}[1]{%
  \expandafter\@arabiceven\csname c@#1\endcsname
}
\newcommand*{\@arabicodd}[1]{%
  \@arabic{\numexpr(#1)*2-1\relax}%
}
\newcommand*{\@arabiceven}[1]{%
  \@arabic{\numexpr(#1)*2\relax}%
}

\AddEnumerateCounter\arabicodd\@arabicodd{0}
\AddEnumerateCounter\arabiceven\@arabiceven{0}
\makeatother

\parindent=0pt
\renewcommand\qedsymbol{Q.E.D.}

\title{Chapter 0 Exercises}
\author{David Piper}

\begin{document}
  \maketitle

  \begin{enumerate}[label=0.\arabicodd*]
    \item
      \begin{enumerate}[label=(\alph*)]
        \item
          \begin{proof}
            Given the functions $f\qty(n)=n^2$ and $g\qty(n)=2n^2+100\sqrt{n}$ we may
            observe that $f=o\qty(g).$ This follows from the fact that $n^2 < 2n^2$
            for all $n \in \mathbb{N}^+.$ To prove this we simply use the definition
            of $f=o\qty(g)$ if for every $\epsilon > 0, f(n) \leq \epsilon \cdot g(n)$
            for every sufficiently large $n.$ From the equation we can observe that $\epsilon=2$
            and that this is true for all $n \geq 1.$ That is, $1^2 < 2 \cdot 1^2$ or
            $1 < 2.$
          \end{proof}
        \item
          \begin{proof}
          \end{proof}
      \end{enumerate}
  \end{enumerate}
\end{document}