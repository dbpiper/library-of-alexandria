\documentclass[12pt]{article}
\usepackage{enumitem, extsizes, amsfonts, braket, amsmath, ragged2e, physics, amsthm }

\parindent=0pt
\renewcommand\qedsymbol{Q.E.D.}

\title{Chapter 0 Exercises}
\author{David Piper}

\begin{document}
  \maketitle

  \begin{enumerate}[label=0.\arabic*]
    \item
      \begin{enumerate}[label=(\alph*)]
        \item \begin{proof}
                Given the functions $f\qty(n)=n^2$ and $g\qty(n)=2n^2+100\sqrt{n}$ we may
                observe that $f=o\qty(g).$ This follows from the fact that $n^2 < 2n^2$
                for all $n \in \mathbb{N}^+.$ To prove this we simply use the definition
                of $f=o\qty(g)$ if for every $\epsilon > 0, f(n) \leq \epsilon \cdot g(n)$
                for every sufficiently large $n.$ From the equation we can observe that $\epsilon=2$
                and that this is true for all $n \geq 1.$ That is, $1^2 < 2 \cdot 1^2$ or
                $1 < 2.$
        \end{proof}
      \end{enumerate}
  \end{enumerate}
\end{document}